%%%%%%%%%%%%%%%%%%%%%%%%%%%%%%%%%%%%%%%%%
% Medium Length Professional CV
% LaTeX Template
% Version 2.0 (8/5/13)
%
% This template has been downloaded from:
% http://www.LaTeXTemplates.com
%
% Original author:
% Trey Hunner (http://www.treyhunner.com/)
%
% Important note:
% This template requires the resume.cls file to be in the same directory as the
% .tex file. The resume.cls file provides the resume style used for structuring the
% document.
%
%%%%%%%%%%%%%%%%%%%%%%%%%%%%%%%%%%%%%%%%%

%----------------------------------------------------------------------------------------
%	PACKAGES AND OTHER DOCUMENT CONFIGURATIONS
%----------------------------------------------------------------------------------------

\documentclass{resume} % Use the custom resume.cls style

\usepackage[left=0.75in,top=0.6in,right=0.75in,bottom=0.6in]{geometry} % Document margins
\usepackage{hyperref}
\usepackage{etaremune}

\name{Ryan Shijie Du} % Your name
\address{Courant Institute of Mathematical Sciences, New York University} % Your secondary addess
\address{251 Mercer Street, New York, NY 10012 \\ \texttt{ryan\_sjdu@nyu.edu}} % Your phone number and email
\address{The Center for Atmosphere Ocean Science (CAOS)} % Your address

\begin{document}

%----------------------------------------------------------------------------------------

\begin{rSection}{Education}

{\bf New York University} \hfill {\em 2020 - Present} \\
\href{https://caos.cims.nyu.edu/dynamic/phd-program/overview}{PhD. in Atmosphere Ocean Science and Mathematics}\\
Advisor: Oliver B\"uhler, Shafer Smith

{\bf University of California, Los Angeles} \hfill {\em 2016 - 2020} \\
B.S. Applied Mathematics (Specializing in Computing); Minor in Philosophy\\
Honors Program in Applied Mathematics, College Honors Program, summa cum laude

\end{rSection}

%----------------------------------------------------------------------------------------

\begin{rSection}{Publications}

%{\bf Refereed}
\begin{etaremune}
    \item \underline{Ryan Shijie Du}, Oliver B\"uhler. \textit{Spectra of the Majda-McLaughlin-Tabak (MMT) model: theory and numerical experiment}. In Preparation.
    \item \href{https://doi.org/10.1103/PhysRevE.104.015109}{\underline{Ryan Shijie Du}, Lily Liu, Simon Ng, Sneha Sambandam, Bernardo Hernandez Adame, Hansell Perez, Kyung Ha, Claudia Falcon, Joseph de Rutte, Dino Di Carlo, Andrea L. Bertozzi. \textit{Statistical energy minimization theory for systems of drop-carrier particles}. Phys. Rev. E 104, 015109 (2021).}
%\end{etaremune}

%{\bf Manuscript}
%\begin{etaremune}
    \item Michael R Lindstrom, \underline{Ryan Shijie Du}, Xiang Yang Ng, Dominic Diaz, Margaret Koulikova, Matthew Nero, Hannah Ross, Sanjay Shukla, Andrea Bertozzi, and P Jeffrey Brantingham. \textit{Using Local Geographic Features to Predict Changes in the Los Angeles Homeless Population}. UCLA CAM Report 19-62 (2019).
\end{etaremune}
\end{rSection}

%----------------------------------------------------------------------------------------

\begin{rSection}{Presentation}
\href{https://meetings.aps.org/Meeting/DFD19/Session/NP05.36}{The 72nd Annual Meeting of the American Physical Society's Division of Fluid Dynamics (DFD)\\
Poster, Modeling Systems of Drop Carrier Particles Through Energy Minimization,  (Nov 2019).}
\end{rSection}

%----------------------------------------------------------------------------------------

\begin{rSection}{Research Experience}

\begin{rSubsection}{1D Wave Turbulence.}{2021-Present}
{Graduate Student Researcher}{CAOS, Courant Institute, NYU\\\underline{Advisor}: Oliver B\"uhler}
\item Explored through numerical experiments the different parameters that affect that spectra of the Majda-McLaughlin-Tabak (MMT) model.
\item Paper in preparation.
\end{rSubsection}

\begin{rSubsection}{Modeling Systems of Drop Carrier Particles Through Energy
Minimization.}{2019-2021}
{Student Researcher}{Applied Math REU Program, UCLA\\\underline{Advisor}: Andrea Bertozzi, Claudia Falcon}

\item Studied the minimal energy/surface configuration of Drop Carrier Particles using calculus of variation.
\item Analytically and experimentally studied the behavior of liquid redistribution between particles.
\item Developed theory predicting mixing time till convergence for systems of particles.
\item First authored a peer-reviewed article about our result.
\end{rSubsection}

\begin{rSubsection}{Predicting Changes in the LA Homeless Population.}{2018-2019}
{Student Researcher}{Department of Mathematics, UCLA\\\underline{Advisor}: Michael Lindstrom}
\item Explored data mining methods like Topic Modeling, Principal Component Analysis, and Cluster Analysis on data about homeless populations and local geographic features in LA.
\item  Constructed machine learning algorithms (in \texttt{Tensorflow}) that predict homeless population in LA from local geographic features.
\item Composed a detailed student research paper and presented research results. 
\end{rSubsection}

\end{rSection}

%----------------------------------------------------------------------------------------

\begin{rSection}{LEADERSHIP AND SERVICE EXPERIENCE}
\begin{rSubsection}{NYU SIAM Student Chapter.}{2020-Now}
{Founding Board Member}{Courant Institute of Mathematical Sciences, NYU}
\item Founded the NYU student Chapter of Society for Industrial and Applied Mathematics (SIAM). I currently act as The Math Department Liaison.
\item Organized events that are accessible to both the undergraduate and graduate student bodies of Courant, to enrich their education experience and foster bonds between them.
\end{rSubsection}

\begin{rSubsection}{Planetary Scale Ocean Circulation Course.}{2020-2021}
{Teaching Assistance}{\href{https://www.worldsciencefestival.com/education/world-science-scholars/}{World Science Scholars program}, World Science Foundation\\\underline{Instructor}: David Holland}
\item World Science Scholars program provides courses for high-school-aged or younger highly talented students that deepen their mathematical knowledge.
\item Assisted in developing and teaching a 3-hours course on the mathematics and physics behind ocean circulations and showcased methods of mathematical analysis, numerical simulation, and lab experiment.
\item Conducted ``ocean gyres in rotating tank'' experiment in \href{http://www.efdlhome.org/}{Environmental Fluid Dynamics Lab, NYU}.
\end{rSubsection}

\begin{rSubsection}{New Student Adviser and New Student Mentor.}{2018}
{\phantom{text}}{\href{http://newstudents.ucla.edu/}{New Student \& Transition Programs, UCLA}}
\item Advised over 150 new undergraduates in curriculum, student services, and personal issues related to transitioning to university.
\item Facilitated group discussions on topics ranging from the academic environment, student diversity, Title IX/sexual violence, alcohol and substance abuse/sexual health, and student services.
\item Conducted presentations including extensive tours of campus and student services.
\end{rSubsection}
\end{rSection}

%----------------------------------------------------------------------------------------

\begin{rSection}{Technical Strengths}

\begin{tabular}{ @{} >{\bfseries}l @{\hspace{6ex}} l }
Languages & English, Chinese (Mandarin)\\
Programming Languages & MATLAB, Python (including FEniCSx, Tensorflow), C++\\
Software & \LaTeX, Inkscape, Qt (in Python and C++), QGIS
\end{tabular}

\end{rSection}

\end{document}
