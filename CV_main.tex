%%%%%%%%%%%%%%%%%%%%%%%%%%%%%%%%%%%%%%%%%
% Medium Length Professional CV
% LaTeX Template
% Version 2.0 (8/5/13)
%
% This template has been downloaded from:
% http://www.LaTeXTemplates.com
%
% Original author:
% Trey Hunner (http://www.treyhunner.com/)
%
% Important note:
% This template requires the resume.cls file to be in the same directory as the
% .tex file. The resume.cls file provides the resume style used for structuring the
% document.
%
%%%%%%%%%%%%%%%%%%%%%%%%%%%%%%%%%%%%%%%%%

%----------------------------------------------------------------------------------------
%	PACKAGES AND OTHER DOCUMENT CONFIGURATIONS
%----------------------------------------------------------------------------------------

\documentclass{resume} % Use the custom resume.cls style

\usepackage[left=0.75in,top=0.6in,right=0.75in,bottom=0.6in]{geometry} % Document margins
\usepackage{hyperref}
\usepackage{etaremune}

\name{Ryan Sh\`iji\'e D\`u} % Your name
\address{Courant Institute of Mathematical Sciences, New York University} % Your secondary addess
\address{251 Mercer Street, New York, NY 10012 \\ \texttt{ryan\_sjdu@nyu.edu}} % Your phone number and email
\address{The Center for Atmosphere Ocean Science (CAOS)} % Your address

\begin{document}

%----------------------------------------------------------------------------------------

\begin{rSection}{Education}

{\bf Courant Institute, New York University} \hfill {2020 - Present} \\
Center for Atmosphere Ocean Science, Courant Institute of Mathematical Sciences\\
Ph.D. in Atmosphere Ocean Science and Mathematics\\
Advisors: Oliver B\"uhler, Shafer Smith

{\bf University of California, Los Angeles} \hfill {2016 - 2020} \\
B.S. Applied Mathematics (Specialization in Computing); Minor in Philosophy\\
Honors Program in Applied Mathematics, College Honors Program, summa cum laude

% Coursework includes: Numerical Analysis, Stochastic Analysis, (Bayesian) Inverse Problems, Turbulence, Convex Optimization, Mathematical Statistics, and High Performance Computing.

\end{rSection}

%----------------------------------------------------------------------------------------

\begin{rSection}{Publications}
%{\bf Refereed}
\begin{etaremune}
    \item \underline{D\`u, R.S.}, B\"uhler, O. \textit{The impact of frequency bandwidth on a one-dimensional model for dispersive wave turbulenc}. Submitted to Journal of Nonlinear Science.
    \item \underline{Du, R.S.}, Liu, L., Ng, S., Sambandam, S., Hernandez Adame, B., Perez, H., Ha, K., Falcon, C., de Rutte, J., Di Carlo, D., Bertozzi, A.L., 2021. \textit{Statistical energy minimization theory for systems of drop-carrier particles}. Phys. Rev. E 104, 015109.
    \item Lindstrom, M.R., \underline{Du, R.S.}, Ng, X.Y., Diaz, D., Koulikova, M., Nero, M., Ross, H., Shukla, S., Bertozzi, A.L., Brantingham, P.J., 2019. \textit{Using local geographic features to predict changes in the Los Angeles homeless population}. UCLA CAM Reports 19-62
\end{etaremune}
\end{rSection}

%----------------------------------------------------------------------------------------

\begin{rSection}{Presentations}
\begin{etaremune}
 \item \textit{SQG\textsuperscript{+1} as a Model for Submesoscale Asymmetry} (Poster)\\
 $\left.\quad\right.$ at FilaChange 2022, August 2022.
 \item \textit{Domain dependence of wave turbulence theory for the Majda-McLaughlin-Tabak (MMT) model} (Poster)\\
 $\left.\quad\right.$ at the 23rd Conference on Atmospheric and Oceanic Fluid Dynamics (AOFD), June 2022;\\
 $\left.\quad\right.$ and the 2022 Gordon Conference: Ocean Mixing, June 2022.
 \item \textit{Modeling systems of drop carrier particles through energy minimization} (Poster) \\
 $\left.\quad\right.$ at the 72nd Annual Meeting of the Division of Fluid Dynamics (APS DFD), Nov 2019.
\end{etaremune}



\end{rSection}

%----------------------------------------------------------------------------------------

\begin{rSection}{Research Experiences}
\begin{rSubsection}{Next order model balanced model for ocean flows.}{2022-Present}
{Graduate Student Researcher}{CAOS, Courant Institute, NYU\\\underline{Advisors}: Oliver B\"uhler, Shafer Smith}
\item Studied the properties of the QG\textsuperscript{+1} model such as vorticity asymmetry in various settings.
\item Simulated primitive equations in Dedalus for comparison.
\item Extended the next order balanced model to other geophysical fluids equations. 
\end{rSubsection}

\begin{rSubsection}{Turbulence spectra of wave turbulence.}{2021-Present}
{Graduate Student Researcher}{CAOS, Courant Institute, NYU\\\underline{Advisor}: Oliver B\"uhler}
\item Developed and numerically tested a new theory for the turbulent spectra of the Majda-McLaughlin-Tabak (MMT) model.
\item Our results resolved the long-standing inconsistency between wave turbulence theory and numerical simulation results with regards to power law spectra in the inertial range.
\end{rSubsection}

\begin{rSubsection}{Lagrangian Filtering for Mean-Wave Separation.}{Summer 2021}
{Graduate Student Researcher}{CAOS, Courant Institute, NYU\\\underline{Advisor}: Shafer Smith}
\item Tested the technique of Lagrangian Filtering for mean-wave separation of geophysical flows by comparing the algorithm output with theoretically known mean flow.
\end{rSubsection}

\begin{rSubsection}{Modeling Systems of Drop Carrier Particles Through Energy
Minimization.}{2019-2021}
{Student Researcher}{Applied Math REU Program, UCLA\\\underline{Advisors}: Andrea Bertozzi, Claudia Falcon}
\item Studied the properties of Drop Carrier Particles, a new experimental tool in biotechnology, through calculus of variation, probability, and lab experiments.
\end{rSubsection}

\begin{rSubsection}{Predicting Changes in the LA Homeless Population.}{2018-2019}
{Student Researcher}{Department of Mathematics, UCLA\\\underline{Advisor}: Michael Lindstrom}
\item Constructed machine learning architecture (in Tensorflow) aiming at predicting changes in homeless population in LA from local geographic features. 
\end{rSubsection}

\end{rSection}

%----------------------------------------------------------------------------------------

\begin{rSection}{Teaching Experiences}
\begin{rSubsection}{Teaching Assistant {\normalfont\textit{at Courant Institute of Mathematical Sciences, NYU}}}{}{}{}
\item Partial Differential Equations (Undergraduate). \hfill Spring 2023
\item Introduction to Fluid Dynamics (Undergraduate). \hfill Spring 2023
\item Numerical Analysis (Undergraduate). \hfill Fall 2022
\end{rSubsection}
\end{rSection}

%----------------------------------------------------------------------------------------

\begin{rSection}{LEADERSHIP AND SERVICE}
\begin{rSubsection}{NYU Applied Math Summer Undergraduate Research Experience (AM-SURE).}{2023}
{Program Co-coordinator}{Courant Institute of Mathematical Sciences, NYU}
\item Working with faculties and post-docs on organizing the summer applied math research experience for undergraduate students from diverse backgrounds.
\item Responsibilities include: selecting program participants, presenting tutorials, advising students' research, and organizing regular research meetings and social events. 
\end{rSubsection}

\begin{rSubsection}{Courant Splash (cSplash).}{2023}
{Lecturer}{Courant Institute of Mathematical Sciences, NYU}
\item Gave a one-hour outreach talk at cSplash 2023 covering the basics of climate change. 
\end{rSubsection}

\begin{rSubsection}{The Mathematical Contest in Modeling (MCM).}{2023}
{Faculty Adviser}{Courant Institute of Mathematical Sciences, NYU}
\item Advised a team of students to prepare for the contest in the basics of mathematical modeling.
\end{rSubsection}

\begin{rSubsection}{NYU SIAM Student Chapter.}{2020-2022}
{Founding Board Member}{Courant Institute of Mathematical Sciences, NYU}
% \item Founded the NYU student Chapter of Society for Industrial and Applied Mathematics (SIAM).
\item Organized events that are accessible to both the undergraduate and graduate student bodies of Courant.
\end{rSubsection}

\begin{rSubsection}{Planetary Scale Ocean Circulation Course.}{2020-2021}
{Teaching Assistant}{World Science Scholars program, World Science Foundation\\\underline{Instructor}: David Holland}
\item Assisted in developing and teaching a 3-hours course on the mathematics and physics behind ocean circulations and showcased methods of mathematical analysis, numerical simulation, and lab experiment.
\item Conducted ``ocean gyres in rotating tank'' experiment in Environmental Fluid Dynamics Lab, NYU.
\end{rSubsection}

\begin{rSubsection}{New Student Adviser and New Student Mentor.}{2018}
{\phantom{text}}{New Student \& Transition Programs, UCLA}
\item Advised over 150 new undergraduates in curriculum, student services, and personal issues related to transitioning to university. 
% \item Facilitated group discussions on topics ranging from the academic environment, student diversity, Title IX/sexual violence, alcohol and substance abuse, and sexual health.
\end{rSubsection}
\end{rSection}

%----------------------------------------------------------------------------------------

\begin{rSection}{Technical Strengths}

\begin{tabular}{ @{} >{\bfseries}l @{\hspace{6ex}} l }
Languages & English, Chinese (Mandarin)\\
Programming Languages & MATLAB, Python (including Dedalus, FEniCS, Tensorflow), C++\\
Software & \LaTeX, Inkscape, Qt (in Python and C++), QGIS
\end{tabular}

\end{rSection}

%----------------------------------------------------------------------------------------

\begin{rSection}{Websites}
\begin{tabular}{ @{} >{\bfseries}l @{\hspace{6ex}} l }
Personal Website & \href{https://sites.google.com/view/ryan-shijie-du}{sites.google.com/view/ryan-shijie-du}\\
GitHub & \href{https://github.com/Empyreal092}{github.com/Empyreal092}
\end{tabular}



\end{rSection}

\end{document}
