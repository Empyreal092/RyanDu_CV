%%%%%%%%%%%%%%%%%%%%%%%%%%%%%%%%%%%%%%%%%
% Medium Length Professional CV
% LaTeX Template
% Version 2.0 (8/5/13)
%
% This template has been downloaded from:
% http://www.LaTeXTemplates.com
%
% Original author:
% Trey Hunner (http://www.treyhunner.com/)
%
% Important note:
% This template requires the resume.cls file to be in the same directory as the
% .tex file. The resume.cls file provides the resume style used for structuring the
% document.
%
%%%%%%%%%%%%%%%%%%%%%%%%%%%%%%%%%%%%%%%%%

%----------------------------------------------------
%	PACKAGES AND OTHER DOCUMENT CONFIGURATIONS
%----------------------------------------------------

\documentclass{resume} % Use the custom resume.cls style

\usepackage[left=0.75in,top=0.6in,right=0.75in,bottom=0.6in]{geometry} % Document margins
\usepackage[hidelinks]{hyperref}
\usepackage{etaremune}
\usepackage{enumitem}

\name{Ryan Sh\`iji\'e D\`u} % Your name
\address{Department of Geophysics, Colorado School of Mines} % Your secondary addess
\address{ \texttt{ryan\_sjdu@nyu.edu }\\ \href{https://sites.google.com/view/ryan-shijie-du}{ \texttt{sites.google.com/view/ryan-shijie-du}}} % Your phone number and email
% \address{The Center for Atmosphere Ocean Science (CAOS)} % Your address

\begin{document}

%----------------------------------------------------

\begin{rSection}{Education}

{\bf New York University, Ph.D.} in Mathematics and Atmosphere Ocean Science \hfill {2020 - 2025} \\
Center for Atmosphere Ocean Science (CAOS), Courant Institute of Mathematical Sciences\\
% Thesis: \textit{Asymptotic corrections to linear models for the physical ocean at the submesoscale and smaller}
Advisors: Oliver B\"uhler, Shafer Smith

{\bf University of California, Los Angeles, B.Sc.} in Applied Mathematics \hfill {2016 - 2020} \\
Honors Program in Applied Mathematics, minor in Philosophy, summa cum laude
% \\Mentor: Andrea Bertozzi

% Coursework includes: Numerical Analysis, Stochastic Analysis, (Bayesian) Inverse Problems, Turbulence, Convex Optimization, Mathematical Statistics, and High Performance Computing.

\end{rSection}

%----------------------------------------------------

\begin{rSection}{Professional Appointments}

\textbf{Post-doctoral fellow}, Colorado School of Mines, Department of Geophysics \hfill {2025 - Present}\\
Mentor: Bia Villas Bôas

\end{rSection}

%----------------------------------------------------
\begin{rSection}{Refereed Publications}
{\bf Submitted}
\renewcommand{\theenumi}{S.\arabic{enumi}}
\begin{etaremune}
    \setcounter{enumi}{5}
    \item \underline{D\`u, R.S.}, Smith K.S., 2025. \textit{Emergent vorticity asymmetry of one and two-layer shallow water system captured by a next-order balanced model}. In revision at Journal of Fluid Mechanics.
    \item \underline{D\`u, R.S.}, Smith K.S., Bühler, O., 2025. \textit{Next-order balanced model captures submesoscale physics and statistics}. In revision at Journal of Physical Oceanography.
\end{etaremune}
{\bf Refereed}
\renewcommand{\theenumi}{R.\arabic{enumi}}
\begin{etaremune}
    \item \underline{D\`u, R.S.}, B\"uhler, O., 2023. \textit{The Impact of Frequency Bandwidth on a One-Dimensional Model for Dispersive Wave Turbulence}. J Nonlinear Sci 33, 81.
    \item \underline{Du, R.S.}, Liu, L., Ng, S., Sambandam, S., Hernandez Adame, B., Perez, H., Ha, K., Falcon, C., de Rutte, J., Di Carlo, D., Bertozzi, A.L., 2021. \textit{Statistical energy minimization theory for systems of drop-carrier particles}. Phys. Rev. E 104, 015109.
\end{etaremune}
{\bf Other}
\renewcommand{\theenumi}{O.\arabic{enumi}}
\begin{etaremune}
    \item \underline{D\`u, R.S.}, 2025. \textit{Asymptotic corrections to linear models for the physical ocean at the submesoscale and smaller}. Ph.D. Thesis, New York University.
    \item Lindstrom, M.R., \underline{Du, R.S.}, Ng, X.Y., Diaz, D., Koulikova, M., Nero, M., Ross, H., Shukla, S., Bertozzi, A.L., Brantingham, P.J., 2019. \textit{Using local geographic features to predict changes in the Los Angeles homeless population}. UCLA CAM Reports 19-62.
\end{etaremune}
\end{rSection}

%----------------------------------------------------

\begin{rSection}{Selected Presentations}
\renewcommand{\theenumi}{P.\arabic{enumi}}
\begin{etaremune}
    \item \textit{Next-order in Rossby SQG\textsuperscript{+1} model for reconstructing velocity from sea surface height}\\
        $\left.\;\right.$ AGU Annual Meeting (AGU24), \hfill 2024
    \item \textit{Next-order balanced model for shallow water captures vorticity asymmetry}\\
        $\left.\;\right.$ AGU Annual Meeting (AGU24), \hfill 2024
    \item \textit{Next-order balanced model captures submesoscale physics and statistics}\\
        $\left.\;\right.$ AGU Ocean Sciences Meeting (OSM24), \hfill 2024
        % \\ $\left.\;\right.$ Conf. on Atmo. and Oceanic Fluid Dynamics (AOFD24), \hfill 2024.
    \item \textit{SQG\textsuperscript{+1} as a Model for Submesoscale Asymmetry}\\
        $\left.\;\right.$ FilaChange Workshop, \hfill 2022
    \item \textit{Domain dependence of wave turbulence theory for the Majda-McLaughlin-Tabak (MMT) model}\\
        $\left.\;\right.$ Conf. on Atmo. and Oceanic Fluid Dynamics (AOFD22), \hfill 2022\\
        $\left.\;\right.$ 2022 Gordon Conference: Ocean Mixing, \hfill 2022
    % \item \textit{Modeling systems of drop carrier particles through energy minimization} \\
    %     $\left.\;\right.$ 72nd Annual Meeting of the Division of Fluid Dynamics (APS DFD), \hfill 2019.
\end{etaremune}
\end{rSection}

%----------------------------------------------------

% \begin{rSection}{Research Experiences}
% \begin{rSubsection}{Next order model balanced model for ocean flows.}{2022-Present}
% {Graduate Student Researcher}{CAOS, Courant Institute, NYU\\\underline{Advisors}: Oliver B\"uhler, Shafer Smith}
% \item Studied the properties of the QG\textsuperscript{+1} model such as vorticity asymmetry in various settings.
% \item Simulated primitive equations in Dedalus for comparison.
% \item Extended the next order balanced model to other geophysical fluids equations. 
% \end{rSubsection}

% \begin{rSubsection}{Turbulence spectra of wave turbulence.}{2021-Present}
% {Graduate Student Researcher}{CAOS, Courant Institute, NYU\\\underline{Advisor}: Oliver B\"uhler}
% \item Developed and numerically tested a new theory for the turbulent spectra of the Majda-McLaughlin-Tabak (MMT) model.
% \item Our results resolved the long-standing inconsistency between wave turbulence theory and numerical simulation results with regards to power law spectra in the inertial range.
% \end{rSubsection}

% \begin{rSubsection}{Lagrangian Filtering for Mean-Wave Separation.}{Summer 2021}
% {Graduate Student Researcher}{CAOS, Courant Institute, NYU\\\underline{Advisor}: Shafer Smith}
% \item Tested the technique of Lagrangian Filtering for mean-wave separation of geophysical flows by comparing the algorithm output with theoretically known mean flow.
% \end{rSubsection}

% \begin{rSubsection}{Modeling Systems of Drop Carrier Particles Through Energy
% Minimization.}{2019-2021}
% {Student Researcher}{Applied Math REU Program, UCLA\\\underline{Advisors}: Andrea Bertozzi, Claudia Falcon}
% \item Studied the properties of Drop Carrier Particles, a new experimental tool in biotechnology, through calculus of variation, probability, and lab experiments.
% \end{rSubsection}

% \begin{rSubsection}{Predicting Changes in the LA Homeless Population.}{2018-2019}
% {Student Researcher}{Department of Mathematics, UCLA\\\underline{Advisor}: Michael Lindstrom}
% \item Constructed machine learning architecture (in Tensorflow) aiming at predicting changes in homeless population in LA from local geographic features. 
% \end{rSubsection}

% \end{rSection}

\begin{rSection}{Awards}
\begin{itemize}
    \item Thomas Tyler Bringley Fellowship, Courant Institute, NYU \hfill 2024\\
    $\left.\;\right.$ for outstanding work in applied mathematics
\end{itemize}
\end{rSection}

\begin{rSection}{Mentoring}
\begin{itemize}
    \itemsep -0.3em
    \item Mentor for Qi Liu \hfill 2024-2025\\
    $\left.\;\right.$ Undergraduate research at NYU, now Ph.D. student at NYU CAOS
    \item Co-mentor for Kai Hung and Daniel Wang \hfill 2023 \\
    $\left.\;\right.$ NYU Applied Math Summer Undergraduate Research Experience (AM-SURE)
    \item Co-mentor for Andreas Louskos  \hfill 2023\\
    $\left.\;\right.$ Master student thesis at NYU
\end{itemize}
\end{rSection}

%----------------------------------------------------

\begin{rSection}{ACADEMIC AND UNIVERSITY SERVICE}
\begin{itemize}
    \itemsep -0.3em
    \item Peer reviewer for \textit{Geophysical Research Letters} (GRL), \textit{Journal of Advances in Modeling Earth Systems} (JAMES), \textit{Journal of Geophysical Research} (JGR:Oceans), 
    \textit{Journal of Atmospheric and Oceanic Technology} (JTECH)
    \item Program co-coordinator for NYU Applied Math Summer Undergraduate Research Experience (AM-SURE) \hfill 2023
    \item Faculty adviser for the Mathematical Contest in Modeling (MCM) \hfill 2023-2024
    \item Member of the committee on reviewing the results of the Courant Ph.D. student survey \hfill 2024
\end{itemize}
\end{rSection}

%----------------------------------------------------
\begin{rSection}{Teaching Experiences}
\begin{itemize}
    % \itemsep -0.3em
    \item Guest lecturer for Columbia's Geophysical Fluid Dynamics course \hfill 2024\\
    $\left.\;\right.$ on using Dedalus to simulate some classic simple models of GFD
    \item Teaching Assistant for Undergraduate courses at Courant, NYU\\
    $\left.\;\right.$ Probability \& Statistics \hfill Spring 2025\\
    $\left.\;\right.$ Analysis \hfill Spring, Fall 2024\\
    $\left.\;\right.$ Numerical Analysis \hfill Fall 2022, 2023\\
    $\left.\;\right.$ Partial Differential Equations \hfill Spring 2023\\
    $\left.\;\right.$ Introduction to Fluid Dynamics \hfill Spring 2023
\end{itemize}
\end{rSection}

%----------------------------------------------------
\begin{rSection}{Outreach}
\begin{itemize}
    \itemsep -0.3em
    \item Lecturer at Courant Splash (cSplash) to NYC high school students \hfill 2025, 2023
    \item Lecturer at NYU College \& Career Lab to rising 8\textsuperscript{th} grade students \hfill 2023
    \item Founding board member of NYU SIAM student chapter \hfill 2020-2022
    \item Teaching assistant of a planetary scale ocean circulation course for the World Science Scholars program, led by Professor David Holland \hfill 2020-2021
    \item New Student Adviser and New Student Mentor at UCLA \hfill 2018
\end{itemize}
\end{rSection}

% \begin{rSection}{ACADEMIC AND UNIVERSITY SERVICE}
% \begin{rSubsection}{NYU Applied Math Summer Undergraduate Research Experience (AM-SURE).}{2023}
% {Program Co-coordinator}{Courant Institute of Mathematical Sciences, NYU}
% \item Working with faculties and post-docs on organizing the summer applied math research experience for undergraduate students from diverse backgrounds.
% \item Responsibilities include: selecting program participants, presenting tutorials, advising students' research, and organizing regular research meetings and social events. 
% \end{rSubsection}

% \begin{rSubsection}{Courant Splash (cSplash).}{2023}
% {Lecturer}{Courant Institute of Mathematical Sciences, NYU}
% \item Gave a one-hour outreach talk at cSplash 2023 covering the basics of climate change. 
% \end{rSubsection}

% \begin{rSubsection}{The Mathematical Contest in Modeling (MCM).}{2023}
% {Faculty Adviser}{Courant Institute of Mathematical Sciences, NYU}
% \item Advised a team of students to prepare for the contest in the basics of mathematical modeling.
% \end{rSubsection}

% \begin{rSubsection}{NYU SIAM Student Chapter.}{2020-2022}
% {Founding Board Member}{Courant Institute of Mathematical Sciences, NYU}
% % \item Founded the NYU student Chapter of Society for Industrial and Applied Mathematics (SIAM).
% \item Organized events that are accessible to both the undergraduate and graduate student bodies of Courant.
% \end{rSubsection}

% \begin{rSubsection}{Planetary Scale Ocean Circulation Course.}{2020-2021}
% {Teaching Assistant}{World Science Scholars program, World Science Foundation\\\underline{Instructor}: David Holland}
% \item Assisted in developing and teaching a 3-hours course on the mathematics and physics behind ocean circulations and showcased methods of mathematical analysis, numerical simulation, and lab experiment.
% \item Conducted ``ocean gyres in rotating tank'' experiment in Environmental Fluid Dynamics Lab, NYU.
% \end{rSubsection}

% \begin{rSubsection}{New Student Adviser and New Student Mentor.}{2018}
% {\phantom{text}}{New Student \& Transition Programs, UCLA}
% \item Advised over 150 new undergraduates in curriculum, student services, and personal issues related to transitioning to university. 
% % \item Facilitated group discussions on topics ranging from the academic environment, student diversity, Title IX/sexual violence, alcohol and substance abuse, and sexual health.
% \end{rSubsection}
% \end{rSection}

%----------------------------------------------------
\begin{rSection}{Technical Strengths}

\begin{tabular}{ @{} >{\bfseries}l @{\hspace{6ex}} l }
Languages & English, Chinese (Mandarin)\\
Programming Languages & MATLAB, Python (including Dedalus, JAX, FEniCS), C++\\
Software & \LaTeX, Inkscape, Qt (in Python and C++), QGIS
\end{tabular}

\end{rSection}

%----------------------------------------------------

% \begin{rSection}{Websites}
% \begin{tabular}{ @{} >{\bfseries}l @{\hspace{6ex}} l }
% Personal Website & \href{https://sites.google.com/view/ryan-shijie-du}{sites.google.com/view/ryan-shijie-du}\\
% GitHub & \href{https://github.com/Empyreal092}{github.com/Empyreal092}
% \end{tabular}
% \end{rSection}

\end{document}
